\documentclass[preprint]{aastex}
%% EXPAND OUR WORKSPACE...
%\setlength{\topmargin}{-0.6in}
%\setlength{\textheight}{9.43in}
%\newcommand{\nts}[1]{{\bf!!!({#1})!!!}}
\begin{document}

\title{USING {\tt tbgfitflex\_exp.pro}}
\author{Carl Heiles; heiles@astro.berkeley.edu: {\bf \today}}

\begin{abstract}
Deriving spin temperatures of self-absorbing HI 21-cm line components
has some subtle aspects involving the background continuum. We discuss
those here and how to incorporate those considerations into the use of
{\tt tbgfitflex\_exp.pro} . Also, this program is totally flexible and
can be used for fitting all cases of Gaussian components for the 21-cm
line. 
\end{abstract}

\section{Radiative Transfer of the 21-cm Line}

Consider a particular velocity of the 21-cm line for which we have,
along the line of sight from near to far:: \begin{enumerate}

\item Nearest, an HI cloud with optical depth $\tau$ and spin
  temperature $T_s$ (meant to be the Cold Neutral Medium, or CNM)

\item Next, an HI cloud with negligible optical depth and huge spin
  temperature (meant to be the Warm Neutral Medium, or WNM) which (in
  the absence of any foreground absorption or background emission)
  produces antenna temperature $T_{W}$

\item Furthest away, continuum emission with antenna temperature
  $T_C$. This might come from any or all of 'pont source', distributed
  Galactic synchrotron emission, and the Cosmic Microwave
  Background. $T_C$ always has the 2.8 K CMB and some (perhaps weak)
  Galactic synchrotron background.
\end{enumerate}

The equation of transfer that specifies the observed antenna temperature
$T_Z$ is

\begin{equation} \label{on}
T_A = T_s \left[1 - \exp(-\tau)\right]\ \ + \ \ [T_W + T_C] \exp(-\tau)
\end{equation}

Normally we perform frequency-switched obseervations. The off-line
antenna temperature is just $T_C$. The switched spectrum is the
difference, $\Delta T_A = online \ - \ offline$:

\begin{equation} \label{diff}
\Delta T_A = \left[T_s - T_C\right] \left[1 - \exp(-\tau)\right] \ \  + \ \ T_W \exp(-\tau)
%\Delta T_A = T_s \left[1 - \exp(-\tau)\right]\ \ + \ \ [T_W + T_C] \exp(-\tau)
\end{equation}

We assume that the observed antenna temperatues versus velocity are
known, and that the $T_C$ is known.  We assume that, for the CNM, the
optical depth profile is given by a Gaussian; and that, for the WNM, the
emission profile is given by a Gaussian. We have many equations like
this, each equation at a different velocity. The unknown parameters that
we solve for are (1) $T_s$; (2) three Gaussian parameters for each CNM
component; (3) three Gaussian parameters for each WNM component.  Our
program {\tt tbgfitflex\_exp} solves for these unknowns.

\section{Running {\tt tbgfitflex\_exp}: Comments, etc.}

In principle, we could use either equation above in this least-squares
solution. In practice, we use equation \ref{on} because it is more
straightforward to deal with softwarewise. Thus, the software expects,
as input, $T_A$ from equation \ref{on}. In contrast, it is often true
that you have available to you the switched spectrum $\Delta T_A$ from
equation \ref{diff}, in which case you must add $T_C$ to that switched
spectrum.  Some comments: \begin{enumerate}

\item Use $T_A$ as the input to the program. This means that the
  `baseline', the portion of the spectrum that lies outside the line,
  must {\it not} be zero, as it is with the switched spectrum $\Delta
  T_A$, but ratheer should equal $T_C$ outside the line. Thus, if you
  have only the switched spectrum $\Delta T_A$, use as the input $T_A =
  \Delta T_A + T_C$. 

\item The quantity $T_A$ is never negative. If your spectrum $T_A =
  \Delta T_A + T_C$ is negative at some velocities (where the CNM
  optical depth is high) then you are using too small a value of $T_C$.

\item $T_C$ consists of the CBR (2.8 K), any point- or
  small-extneded-source contributions, and the Galactic synchrotron
  background. To estimate this Galactic contribution: \begin{enumerate}

\item Go to Haslam, Salter, Stoffel, \& Wilson (1982 A\&A Suppl 47, 1:
  `A 408 MHz all-sky continuum survey. II - The atlas of contour
  maps'), read off the 408 MHz brightness temperature (or use the
  electronic catalog version) $T_{408}$. 

\item The Galactic synchrotron background has a brightness temperature
  spectral index of about 2.7. Thus, divide the $T_{408}$ by $\left({1420
  \over 408}\right)^{2.7}$, i.e.\ divide by 30. 

\item When fitting Gaussians, my experience is that it's a good idea to
  allow for a floating zero offset. That is, allow the program to solve
  for the continuum zero point by setting {\tt continuum\_yn = 1}. This
  will change the value of {\tt continuum} a little bit, which won't
  matter much for derived spin temperatures or oher quantities, but it
  will probably help the fit to converge and provide better Gaussian-fit
  parameters, particularly for wide lines.

\end{enumerate}

\item Stokes I is the sum of two orthogonal polarizations. If you are
  fitting a Stokes I profile, realize the following: \begin{enumerate}

\item The line profile is Stokes I, which is twice the normal
  single-polarization value.

\item Thus, divide the Stokes I intensities by 2 for the input to this
  program. 
\end{enumerate}
\end{enumerate}

\section{{\tt tbgfitflex\_exp} Works for Both Absorption and Emission Lines---It's Totally Flexible!}

\subsection{The CNM and the WNM}

{\tt tbgfitflex\_exp} is {\it totally flexible!} It assumes two kinds of
21-cm line.  The naming convention is based on the HI being either CNM
or WNM, where the CNM components absorb and also emit, while the WNM
components are optically thin and have no absorption. If you molecules,
such as OH, the optical depth for any component is probably always
nonzero, so you can treat all components as being CNM. To be more
specific: \begin{enumerate}

\item Components with Gaussian parameters designated by the prefix {\tt
  cnm} (`Cold Neutral Medium', or `CNM') have nonzero optical depth and
  (1) absorb the radiation coming from behind, as well as (2) emit their
  own radiation. Each Gaussian compoment has a central optical depth,
  center, width, and spin temperature. Additionally, you have to specify
  the order along the line of sight ({\tt order}), because nearer clouds
  absorb the radiation from more distant ones.

\item Components with Gaussian parameters designated by the prefix {\tt
  wnm} (`Warm Neutral Medium', or `WNM'), have zero optical depth. Each WNM
  component can be placed anywhere with respect to all the CNM
  components, specified by the parameter {\tt fwnm}: {\tt fwnm}=0 means
  the WNM lies behind all of the CNM components, {\tt fwnm}=1 means it
  lies in front of all the CNM componnts, and a fractional values
  between 0 and 1 have the obvious meaning and are allowed.

\item The program requires at least one component of each type. This
  does not limit its flexibility: for example, 
if you have only emission lines and optical depths are negligible, then
specify a single, exceedingly weak CNM component whose center lies outside the velocity
range being analyzed and set all the `yn' parameters for this component
equal to zero so that the program doesn't try to least-squares fit the
component. If you have only absorption lines, then do the same for a
single WNM component.

\item You also specify the {\tt continuum}, which is assumed to lie behind all
of the CNM components. If you are fitting an absorption line of a
continuum source (onsrc-offsrc), the {\tt contiinuum} should be the
continuum source deflection.
\end{enumerate}

\section{Gaining Confidence in {\tt tbgfitflex\_exp}}

In the subdirectory that contains this procedure there is an IDL command
file that generates spectra and then does the least squares solution
using {\tt tbgfitflex\_exp}. It is called {\tt tst\_tbg.idl}, invoked
within IDL by typing {\tt @tst\_tbg.idl}. 
\end{document}
